% Template adapted from https://github.com/jgm/pandoc-templates/blob/master/default.latex
% To be used with XeLaTex in memoiR
%%%%%%%%%%%%%%%%%%%%%%%%%%%%%%%%%%%%%%%%%%%%%%%%%%%%%%%%%%%%%%%%%%%%%%%%%%%%%%%%%%%%%%%%%

% Options for packages loaded elsewhere
\PassOptionsToPackage{unicode=true}{hyperref}
\PassOptionsToPackage{hyphens}{url}
\PassOptionsToPackage{dvipsnames,svgnames*,x11names*}{xcolor}
% Right to left support


\documentclass[
  12pt,
  english,
  a4paper,
  extrafontsizes,onecolumn,openright
  ]{memoir}

% Double (or whatever) spacing

% Math
\usepackage{amssymb, amsmath}
% mathspec: arbitrary math fonts
\usepackage{unicode-math}
\defaultfontfeatures{Scale=MatchLowercase}
\defaultfontfeatures[\rmfamily]{Ligatures=TeX,Scale=1}

% Fonts
\usepackage{lmodern}
\usepackage{fontspec}
% Main font
% Specific sanserif font
% Specific monotype font
% Specific math font
% Chinese, Japanese, Corean fonts

% Use upquote for straight quotes in verbatim environments
\usepackage{upquote}
% Use microtype
\usepackage[]{microtype}
\UseMicrotypeSet[protrusion]{basicmath} % disable protrusion for tt fonts

% Verbatim in note

% Color links
\usepackage{xcolor}

% Strikeout

% Necessary for code chunks

% Listings package

% Tables
\usepackage{longtable,booktabs,tabu}
% Fix footnotes in tables (requires footnote package)
\IfFileExists{footnote.sty}{\usepackage{footnote}\makesavenoteenv{longtable}}{}

% Graphics
\usepackage{graphicx,grffile}
\graphicspath{{images/}}
\makeatletter
\def\maxwidth{\ifdim\Gin@nat@width>\linewidth\linewidth\else\Gin@nat@width\fi}
\def\maxheight{\ifdim\Gin@nat@height>\textheight\textheight\else\Gin@nat@height\fi}
\makeatother
% Scale images if necessary, so that they will not overflow the page
% margins by default, and it is still possible to overwrite the defaults
% using explicit options in \includegraphics[width, height, ...]{}
\setkeys{Gin}{width=\maxwidth,height=\maxheight,keepaspectratio}

% Prevent overfull lines
\setlength{\emergencystretch}{3em}  
\providecommand{\tightlist}{%
  \setlength{\itemsep}{0pt}\setlength{\parskip}{0pt}}

% Number sections for memoir (secnumdepth counter is ignored)
\setsecnumdepth{section}

% Set default figure placement to htbp
\makeatletter
\def\fps@figure{htbp}
\makeatother

% Spacing in lists
\usepackage{enumitem}

% Polyglossia
\usepackage{polyglossia}
\setmainlanguage{en}

% localized quotes
\usepackage[strict,autostyle]{csquotes}

% BibLaTeX
\usepackage[backend=biber,style=authoryear-ibid,isbn=false,backref=true,giveninits=true,uniquename=init,maxcitenames=2,maxbibnames=150,sorting=nyt,sortcites=false]{biblatex}
\addbibresource{references.bib}

% cslreferences environment required by pandoc > 2.7



%%%%%%%%%%%%%%%%%%%%%%%%%%%%%%%%%%%%%%%%%%%%%%%%%%%%%%%%%%
% memoiR format

% Chapter Summary environment 
\usepackage[tikz]{bclogo}
\newenvironment{Summary}
  {\begin{bclogo}[logo=\bctrombone, noborder=true, couleur=lightgray!50]{Shrnutí v kostce}\parindent0pt}
  {\end{bclogo}}
% Syntax:
%
%```{block, type='Summary'}
% Deliver message here.
% ```

% scriptsize code 
\let\oldverbatim\verbatim
\def\verbatim{\oldverbatim\scriptsize}
% Applies to code blocks and R code results
% code chunk options size='scriptsize' applies only to R code and results
% if the code chunk sets a different size, \def\verbatim{...} is prioritary for code results 


% memoiR dalef3 chapter style 
% https://ctan.crest.fr/tex-archive/info/latex-samples/MemoirChapStyles/MemoirChapStyles.pdf
\usepackage{soul}
\definecolor{nicered}{rgb}{.647,.129,.149}
\makeatletter
\newlength\dlf@normtxtw
\setlength\dlf@normtxtw{\textwidth}
\def\myhelvetfont{\def\sfdefault{mdput}}
\newsavebox{\feline@chapter}
\newcommand\feline@chapter@marker[1][4cm]{%
  \sbox\feline@chapter{%
    \resizebox{!}{#1}{\fboxsep=1pt%
	  \colorbox{nicered}{\color{white}\bfseries\sffamily\thechapter}%
	}}%
  \rotatebox{90}{%
    \resizebox{%
	  \heightof{\usebox{\feline@chapter}}+\depthof{\usebox{\feline@chapter}}}%
	{!}{\scshape\so\@chapapp}}\quad%
  \raisebox{\depthof{\usebox{\feline@chapter}}}{\usebox{\feline@chapter}}%
 }
\newcommand\feline@chm[1][4cm]{%
  \sbox\feline@chapter{\feline@chapter@marker[#1]}%
  \makebox[0pt][l]{% aka \rlap
    \makebox[1cm][r]{\usebox\feline@chapter}%
  }}
\makechapterstyle{daleif1}{
  \renewcommand\chapnamefont{\normalfont\Large\scshape\raggedleft\so}
  \renewcommand\chaptitlefont{\normalfont\huge\bfseries\scshape\color{nicered}}
  \renewcommand\chapternamenum{}
  \renewcommand\printchaptername{}
  \renewcommand\printchapternum{\null\hfill\feline@chm[2.5cm]\par}
  \renewcommand\afterchapternum{\par\vskip\midchapskip}
  \renewcommand\printchaptertitle[1]{\chaptitlefont\raggedleft ##1\par}
}
\makeatother


% Layout
%%%%%%%%%%%%%%%%%%%%%%%%%%%%%%%%%%%%%%%%%%%%%%%%%%%%%%%%%%

% Based on memoir, style companion
\newcommand{\MemoirChapStyle}{daleif1}
\newcommand{\MemoirPageStyle}{Ruled}

% Space between paragraphs
\usepackage{parskip}
  \abnormalparskip{3pt}

% Adjust margin paragraphs vertical position
\usepackage{marginfix}


% Margins
%%%%%%%%%%%%%%%%%%%%%%%%%%%%%%%%%%%%%%%

% allow use of '-',+','/' ans '*' to make simple length computation
\usepackage{calc}

% Full-width figures utilities
\newlength\widthw % full width
\newlength{\rf}
\newcommand*{\definesHSpace}{
  \strictpagecheck % slower but efficient detection of odd/even pages
  \checkoddpage
  \ifoddpage
  \setlength{\rf}{0mm}
  \else
  \setlength{\rf}{\marginparsep+\marginparwidth}
  \fi
}

\makeatletter
% 1" margins for the front matter.
\newcommand*{\SmallMargins}{
  \setlrmarginsandblock{1.5in}{1.5in}{*}
  \setmarginnotes{0.1in}{0.1in}{0.1in}
 \setulmarginsandblock{1.5in}{1in}{*}
  \checkandfixthelayout
  \ch@ngetext
  \clearpage
  \setlength{\widthw}{\textwidth+\marginparsep+\marginparwidth}
  \footnotesatfoot
  \chapterstyle{\MemoirChapStyle}  % Chapter and page styles must be recalled
  \pagestyle{\MemoirPageStyle}
}

% 3" outer margin for the main matter
\newcommand{\LargeMargins}{\SmallMargins}
\makeatother

% Figure captions and footnotes in outer margins


% Local toc
%%%%%%%%%%%%%%%%%%%%%%%%%%%%%%%%%%%%%%%%%%%%%%%%%%%%%%%%%%

\usepackage{titletoc}
\newcommand{\toc}[1]{%
  \startcontents[chapters]%
  \printcontents[chapters]{}{1}[#1]{}%
  ~\newline%
}


% Text boxes
%%%%%%%%%%%%%%%%%%%%%%%%%%%%%%%%%%%%%%%%%%%%%%%%%%%%%%%%%%

% Define a style for mdframed boxes
\mdfdefinestyle{boxstyle}{
	skipabove=1.5\topskip,
	skipbelow=.5\topskip,
	rightmargin=0pt,
	leftmargin=0pt,
	innerrightmargin=7pt,
	innerleftmargin=7pt,
	topline=false,
	bottomline=false,
	rightline=false,
	leftline=false,
	frametitlerule=true,
	linecolor=black,
	fontcolor=black,
	frametitlealignment=\noindent
}


% Main title page with filigrane
%%%%%%%%%%%%%%%%%%%%%%%%%%%%%%%%%%%%%%%%%%%%%%%%%%%%%%%%%%

% Text blocks
\usepackage[absolute,overlay]{textpos}
  \setlength{\TPHorizModule}{1mm}
  \setlength{\TPVertModule}{1mm}

\newcommand{\MainTitlePage}[2]{
  \SmallMargins % Margins
  \pagestyle{empty} % No header/footer
  \textblockorigin{\stockwidth-\paperwidth-\trimedge}{\trimtop} % recto
  \begin{textblock*}{2mm}(\spinemargin/2,\uppermargin/2)
    \rule{1pt}{\paperheight-\uppermargin}
  \end{textblock*}
  \begin{textblock*}{\paperwidth*2/3}(\paperwidth/5, \paperheight/5)
    \flushright
    \begin{Spacing}{3}
      {\fontfamily{qtm}\selectfont\fontsize{45}{45}\selectfont\textsc{\thetitle}}
    \end{Spacing}
  \end{textblock*}
    \begin{textblock*}{\paperwidth*2/3}(\paperwidth/5, \paperheight/2)
    \flushright
    {\fontfamily{qtm}\huge\theauthor}
  \end{textblock*}
    \begin{textblock*}{\paperwidth*2/3}[0, 1](\spinemargin, \uppermargin+\textheight)
    \normalfont\thedate
  \end{textblock*}
  ~\\ % Print a character or the page will not exist
  \newpage
  \textblockorigin{\trimedge}{\trimtop} % verso
  \begin{textblock*}{\textwidth}(\paperwidth-\spinemargin-\textwidth, \uppermargin)
    #1
  \end{textblock*}
  \begin{textblock*}{\textwidth}[0,1](\paperwidth-\spinemargin-\textwidth, \uppermargin+\textheight+\footskip)
    \centering
          \includegraphics[width=\paperwidth/4]{logo}\\ \bigskip
        #2
  \end{textblock*}
  ~\\ % Print a character or the page will not exist
  \newpage
}

% Clear page and open an even one (\clearpage opens an odd one)
\newcommand{\evenpage}{
  \clearpage
  \strictpagecheck % slower but efficient detection of odd/even pages
  \checkoddpage
  \ifoddpage
    \thispagestyle{empty}
    ~\\ % Print a character or the page will not exist
    \newpage
  \else
    % do nothing
  \fi
}


%% PDF title page to insert
%%%%%%%%%%%%%%%%%%%%%%%%%%%%%%%%%%%%%%%%%%%%%%%%%%%%%%%%%%

\usepackage{pdfpages}


%% Bibliography
%%%%%%%%%%%%%%%%%%%%%%%%%%%%%%%%%%%%%%%%%%%%%%%%%%%%%%%%%%

% Repeated citation as author-year-title instead of author-title (modification of footcite:note in verbose-inote.cbx)

%% Table of Contents
%%%%%%%%%%%%%%%%%%%%%%%%%%%%%%%%%%%%%%%%%%%%%%%%%%%%%%%%%%

% fix the typesetting of the part number
\renewcommand\partnumberlinebox[2]{#2\ ---\ }


% Fonts
%%%%%%%%%%%%%%%%%%%%%%%%%%%%%%%%%%%%%%%%%%%%%%%%%%%%%%%%%%


% Hyperref comes last
%%%%%%%%%%%%%%%%%%%%%%%%%%%%%%%%%%%%%%%%%%%%%%%%%%%%%%%%%%

\usepackage{hyperref}
\hypersetup{
  pdftitle={Judovo ověřování historického myšlení},
  pdfauthor={Juda Kaleta},
  colorlinks=true,
  linkcolor=Maroon,
  citecolor=Blue,
  urlcolor=blue,
  breaklinks=true}

% Don't use monospace font for urls
\urlstyle{same}


% Title, author, date from YAML to LaTeX
%%%%%%%%%%%%%%%%%%%%%%%%%%%%%%%%%%%%%%%%%%%%%%%%%%%%%%%%%%

\title{Judovo ověřování historického myšlení}

\author{Juda Kaleta}

\date{2026-02-20}


% Include headers (preamble.tex) here
%%%%%%%%%%%%%%%%%%%%%%%%%%%%%%%%%%%%%%%%%%%%%%%%%%%%%%%%%%
% Add LaTeX code into the preamble of the document here
\hyphenation{bio-di-ver-si-ty sap-lings}

% Define colors for text boxes
\definecolor{grey}{HTML}{F5F5F5}

% Define text box environments
\newmdenv[
	style=boxstyle,
	backgroundcolor=grey,
	frametitlebackgroundcolor=grey,
]{greybox}
\usepackage{booktabs}
\usepackage{longtable}
\usepackage{array}
\usepackage{multirow}
\usepackage{wrapfig}
\usepackage{float}
\usepackage{colortbl}
\usepackage{pdflscape}
\usepackage{tabu}
\usepackage{threeparttable}
\usepackage{threeparttablex}
\usepackage[normalem]{ulem}
\usepackage{makecell}
\usepackage{xcolor}


% End of preamble
%%%%%%%%%%%%%%%%%%%%%%%%%%%%%%%%%%%%%%%%%%%%%%%%%%%%%%%%%%


\begin{document}
\frontmatter

% Title page
%%%%%%%%%%%%%%%%%%%%%%%%%%%%%%%%%%%%%%%%%%%%%%%%%%%%%%%%%%

\includepdf[pages=1]{images/cover.pdf}
\cleardoublepage

\MainTitlePage{This document is reproducible thanks to:

\begin{itemize}
  \item \LaTeX and its class memoir (\url{http://www.ctan.org/pkg/memoir}).
  \item R (\url{http://www.r-project.org/}) and RStudio (\url{http://www.rstudio.com/})
  \item bookdown (\url{http://bookdown.org/}) and memoiR (\url{https://ericmarcon.github.io/memoiR/})
\end{itemize}}{Name of the owner of the logo

\url{http://www.company.com}

An explanatory sentence.
Leave an empty line for line breaks.}


% Before Body
%%%%%%%%%%%%%%%%%%%%%%%%%%%%%%%%%%%%%%%%%%%%%%%%%%%%%%%%%%





% Contents
%%%%%%%%%%%%%%%%%%%%%%%%%%%%%%%%%%%%%%%%%%%%%%%%%%%%%%%%%%

\LargeMargins
{
\hypersetup{linkcolor=}
\setcounter{tocdepth}{2}
\tableofcontents
}


% Body
%%%%%%%%%%%%%%%%%%%%%%%%%%%%%%%%%%%%%%%%%%%%%%%%%%%%%%%%%%

\LargeMargins
\chapter*{Úvod}\label{uxfavod}
\addcontentsline{toc}{chapter}{Úvod}

Toto je veřejné sdílení předběžných zjištění, myšlenek a materiálů k diskuzi v rámci mého doktorského studia,
ve kterém se zaměřuji na problematiku historického myšlení a jeho ověřování ve výuce dějepisu.

Cíle tohoto webu jsou následující:

\begin{enumerate}
\def\labelenumi{\arabic{enumi}.}
\tightlist
\item
  Protože samotná dizertace i průběžné výstupy budou převážně v angličtině, tento web by měl zjištění zprostředkovat i českému čtenáři.
\item
  Zároveň ve shodě s principy open science se budu snažit zprostředkovávat průběžný stav výzkumu -- a otevřít ho tak přípomínkám, dotazům a návrhům na vylepšení.
\item
  V neposlední řádě chci touto cestou vrátit svůj dluh zapojeným učitelům a sdílet s nimi materiály, které by sami mohli adaptovat v hodinách.
\end{enumerate}

Z principu je tento web \emph{work-in-progress} - jeho obsah se bude proměňovat. Veškeré změny jsou evidovány v logu \href{https://github.com/yetty/phd-public}{Github repozitáře}.

Obsah neprochází žádnou formální kontrolou, ani po stránce obsahové, ani jazykové. Pokud narazíte na (třeba jazykovou) chybu, alespoň víte, že nečtete žádný výplod umělé inteligence.

V každém případě mi můžete dát vědět na mail \href{mailto:juda.kaleta@gmail.com}{\nolinkurl{juda.kaleta@gmail.com}}.

Díky a ať je vám obsah k užitku.

\begin{center}\rule{0.5\linewidth}{0.5pt}\end{center}

\emph{Veškerý obsah tohoto webu je publikován pod licencí CC BY 4.0}

\chapter{HPT a extremismus}\label{hpt-a-extremismus}

Existující modely historického myšlení pracují s konceptem tzv. dobových perspektiv (\emph{historical perspectives}). Projekt Dějepis+ je rámoval otázkou \emph{\enquote{Jak lépe porozumět lidem, kteří žili v minulosti?}} \autocite[ p.~10]{Cinatl2021Metodika}. Jak ale, že žákům se porozumět lidem v minulosti daří?

Jedním z nástrojů, který se o to pokouší, je tzv. Historical Perspective Taking (HPT) dotazník \autocite{Hartmann_Hasselhorn2008Historical}. Má solidní empirické podložení a úspěšně ho adaptovaly i další výzkumy \autocite{Huijgen_et_al2014Testing,Huijgen_et_al2017Toward}.

Zdá se mi ale, že má jeden zásadní nedostatek -- nijak nezohledňuje současné perspektivy žáků. Jinými slovy: žák, který dosáhne lepšího výsledku v HPT, nemusí ve skutečnosti vykazovat lepší porozumění lidem v minulosti. Může výsledku dosáhnout proto, že nepřejímá implicitní předpoklad současné perspektivy HPT -- tedy že současná perspektiva všech žáků je \emph{\enquote{nacismus byl špatný}}.

Jak výsledky ovlivňují právě názory (a perspektiva) žáka, bych chtěl ověřit následujícím výzkumem.

\section{Plánovaný výzkum}\label{pluxe1novanuxfd-vuxfdzkum}

\subsection{Plán a cíle výzkumu}\label{pluxe1n-a-cuxedle-vuxfdzkumu}

\textbf{Hlavní cíl:} Ověřit, jestli a jak moc mohou dnešní postoje žáků (příklon k autoritarismu, normalizace nacismu) ovlivnit výsledky ve validovaném nástroji na měření historických perspektiv -- tedy budit dojem dobrého historického myšlení jen proto, že jejich názory souzní s názory tehdejších aktérů.

Kromě toho by výzkum měl čéskym učitelům přispět

\begin{itemize}
\tightlist
\item
  ověřeným českým překladem HPT dotazníku;
\item
  ověřeným nástroje pro ověření současných perspektiv žáků (FR-LF, KSA-3, SDR-5)
\end{itemize}

Tyto nástroje by měly učitelům pomoci lépe diagnostikovat, jak na tom jejich žáci jsou jak se schopností zaujmout dobové perspektivy, tak ohledně příklonu k různým autoritářským postojům -- a v konečném důsledku tedy pomoci lépe přizpůsobit výuku moderních a soudobých dějin potřebám konkrétní třídy.

\subsubsection{Jak to bude prakticky probíhat}\label{jak-to-bude-prakticky-probuxedhat}

\textbf{Kdo:} 9. ročník ZŠ + všechny ročníky SŠ

\textbf{Čas:} 15-25 minut čistého času

\textbf{Jednotlivé části:}

\begin{enumerate}
\def\labelenumi{\arabic{enumi}.}
\tightlist
\item
  \textbf{Sběr základních údajů:} dotazník je anonymní, žáci pouze zadají svou třídu, pohlaví a poslední známku z dějepisu.
\item
  \textbf{Mini-test znalostí (6 položek)} -- pro zjištění vlivu znalostí na schopnost zaujmout dobové perspektivy. \autocite{Huijgen_et_al2017Toward}
\item
  \textbf{Úkol na dobové perspektivy} -- český překlad ověřeného nástroje, který je postaven na fiktiní situaci z meziválečného Německa. \autocite{Hartmann_Hasselhorn2008Historical}
\item
  \textbf{Krátký anonymní dotazník postojů} -- obsahuje náhodně seřazené položky tří nástrojů:

  \begin{itemize}
  \tightlist
  \item
    FR-LF (mini verze o 6 položkách) - měří postoje k autoritářské vládě a zlehčování nacismu \autocite{Decker_et_al2013Rechtsextremismus}
  \item
    KSA-3 -- měří ultra-pravicovou názorovou orientaci \autocite{Beierlein_et_al2014Die_Kurzskala,Nießen_et_al2019An_English}
  \item
    SDR-5 -- měří sociální žádoucnost, tedy jak moc ovlivňuje odpovědi žáků to, aby vypadali dobře \autocite{Hays_et_al1989A_Five}
  \end{itemize}
\end{enumerate}

\textbf{Co bude dělat učitel?}

\begin{itemize}
\tightlist
\item
  zašlete odkaz na vyplnění dotazníku (elektronicky) \emph{nebo} rozdáte vytištěné dotazníky v papírové variantě
\item
  dohlédnete na klidný průběh a odevzdání
\end{itemize}

\textbf{Etika a bezpečí}

\begin{itemize}
\tightlist
\item
  Účast je \textbf{dobrovolná a anonymní}, žák může vyplňování odmítnout -- nenuťte ho, prosím
\item
  Pokud by některé otázky vyvolaly reakci žáků, doporučuji zařadit krátký bezpečný prostor pro reflexi a sdilení
\end{itemize}

\subsubsection{Co jako jako zapojený učitel dostanu?}\label{co-jako-jako-zapojenuxfd-uux10ditel-dostanu}

\begin{enumerate}
\def\labelenumi{\arabic{enumi}.}
\tightlist
\item
  \textbf{Souhrnný report za třídu/školu}:

  \begin{itemize}
  \tightlist
  \item
    jaké žáci prokázali schopnosti zaujmout dobovou perspektivu
  \item
    jaký se u nich projevil vztah mezi znalostmi a schopností zaujmout dobovou perspektivu
  \item
    případné náznaky nežádoucího \enquote{přilepšení} výkonu na základě postojů žáku
  \end{itemize}
\item
  \textbf{Doporučení do výuky}:

  \begin{itemize}
  \tightlist
  \item
    tipy na základě výzkumů, jak prohlubovat schopnost žáků zaujmout dobové perspektivy;
  \item
    tipy, jak vybírat kontext, který žákům pomůže (časový, prostorový, socio-politický, socio-kulturní)
  \end{itemize}
\end{enumerate}

\subsection{Předpokládaná zjištění}\label{pux159edpokluxe1danuxe1-zjiux161tux11bnuxed}

\subsubsection{Scénář A -- potvrzení vlivu postojů žáka}\label{scuxe9nuxe1ux159-a-potvrzenuxed-vlivu-postojux16f-ux17euxe1ka}

Pracovní hypotéza je, že příklon žáka k extremistickým postojům může kontaminovat výsledky testu na schopnost zaujmout dobové perspektivy. Pokud se to potvrdí, bude to pro nás:

\begin{enumerate}
\def\labelenumi{\arabic{enumi}.}
\tightlist
\item
  signál, že při výuce dějepisu je důležité s postoji žáka pracovat a vědomě je reflektovat, protože mohou ovlivnit nejen jeho přemýšlení o minulosti, ale i to, jak jeho výstupy působí skrytě na učitele;
\item
  solidní základ pro přemýslení o tom, jak konstruovat ověřování schopnosti zaujmout dobové perspektivy, které by nebylo náchylné na ovlivnění postoji žáka.
\end{enumerate}

\subsubsection{Scénář B -- vliv postojů žáka se neprokáže}\label{scuxe9nuxe1ux159-b-vliv-postojux16f-ux17euxe1ka-se-neprokuxe1ux17ee}

Pokud se hypotéza vlivu postojů žáka na výsledky neprokáže, bude to trochu zklamání a překvapení. Nicméně i to bude cenné zjištění -- a hlavně, zůstane nám validovaný a funkční nástroj, pomocí kterého bude možné schopnost žáků zaujmout dobovou perspektivu ověřovat.

\subsubsection{FAQ pro učitele}\label{faq-pro-uux10ditele}

\textbf{Není to politický průzkum?} -- Ne. Nezajímají mne přímo politické postoje žáků, ale to, jak mohou ovlivnit (kontaminovat) ověřování schopnosti zaujmout dobovou prespektivu.

\textbf{Proč se pracuje s \enquote{fiktivní} situací bez primárního pramene?} -- Autoři původního nástroje HPT vycházeli z předpokladu, že při použití primárního pramene by výsledky mohla do velké míry ovlivnit schopnost žáků číst dobový pramen.

\textbf{Můžu úlohy a dotazníky nějak připomínkovat?} -- Rozhodně! Pokud máte návrh na vylepšení překladu nebo formulace, napište mi na \href{mailto:juda.kaleta@gmail.com}{\nolinkurl{juda.kaleta@gmail.com}}. Zároveň ale platí, že se snažím být věrný originálu -- takže vylepšení, které se budou příliš vzdalovat od původní verze, sice ocením a poznamenám si, ale spíše nebudu moci začlenit.

\section{Výsledky výzkumu}\label{vuxfdsledky-vuxfdzkumu}

V dějepise se často snažíme žákům pomoci porozumět lidem v minulosti v jejich dobovém kontextu, nikoli je hodnotit dnešníma očima. Tuto schopnost se výzkumně snaží zachytit test Historical Perspective Taking (HPT). Jenže\ldots{}

\phantomsection\label{pythbox}
Nemůže žák dosáhnout dobrého výsledku v testu dobových perspektiv spíše proto, že jeho \textbf{současné postoje souzní s postoji historických aktérů}, než proto, že skutečně dobře historicky uvažuje?

Konkrétně: nemohou zda autoritářské nebo pro-nacistické postoje „vylepšit`` výsledek v úloze zaměřené na porozumění meziválečnému Německu?

\phantomsection\label{pythbox}
\textbf{Ne, předpoklad, že žáci s autoritářskými postoji dosáhnou lepších výsledků v HPT testu, se nepotvrdil.} Korelace mezi postoji a výsledky v HPT byla velmi slabá až nulová. To znamená, že výsledky v HPT testu nejsou kontaminovány současnými postoji žáků. Dobrá zpráva pro učitele dějepisu - výsledky v HPT testu skutečně odrážejí schopnost žáků zaujmout dobové perspektivy, nikoliv jejich současné názory. \textbf{Můžete ho tedy bez obav používat jako nástroj pro diagnostiku historického myšlení vašich žáků.}

\subsection{Jak výzkum probíhal?}\label{jak-vuxfdzkum-probuxedhal}

Výzkum proběhl v prosinci 2025 a zúčastnili se ho žáci 9. ročníků ZŠ a všech ročníků SŠ z různých typů škol a regionů ČR. Cílem při tom nebyla reprezentativnost (výsledky tedy nic neříkají o tom, jak jsou na tom \enquote{čeští žáci} obecně), ale jen zjištění, zda se potvrdí hypotéza, že autoritářské postoje žáků mohou ovlivnit výsledky v HPT testu.

\scriptsize

\normalsize

Celkem se výzkumu zúčastnilo 293 žáků z 10 škol (20 tříd).

\scriptsize

\normalsize

\subsubsection{Podrobnosti o vzorku žáků}\label{podrobnosti-o-vzorku-ux17euxe1kux16f}

\scriptsize

\begin{center}\includegraphics[width=0.8\linewidth]{judovo-overovani-historickeho-mysleni_files/figure-latex/demographics_plots-1} \end{center}

\normalsize

\subsubsection{Co žáci vyplňovali?}\label{co-ux17euxe1ci-vyplux148ovali}

Žáci vyplňovali dotazník v délce přibližně 20-30 minut pomocí Google Forms (online). Dotazník obsahoval:

\begin{enumerate}
\def\labelenumi{\arabic{enumi}.}
\tightlist
\item
  \textbf{Základní údaje} -- konkrétně třída, pohlaví a poslední známka z dějepisu.
\item
  \textbf{Mini-test znalostí} -- 6 otázek na základní znalosti o meziválečném Německu pro kontrolu vlivu znalostí na výsledky HPT.
\item
  \textbf{HPT dotazník} -- český překlad ověřeného nástroje na měření schopnosti zaujmout dobové perspektivy \autocite{Hartmann_Hasselhorn2008Historical}.
\item
  \textbf{Dotazník postojů} -- obsahoval položky z následujících nástrojů:

  \begin{itemize}
  \tightlist
  \item
    FR-LF (mini verze o 6 položkách) -- měří postoje k autoritářské vládě a zlehčování nacismu \autocite{Decker_et_al2013Rechtsextremismus}
  \item
    KSA-3 -- měří ultra-pravicovou názorovou orientaci \autocite{Beierlein_et_al2014Die_Kurzskala,Nießen_et_al2019An_English}
  \item
    SDR-5 -- měří sociální žádoucnost, tedy jak moc ovlivňuje odpovědi žáků to, aby vypadali dobře \autocite{Hays_et_al1989A_Five}
  \end{itemize}
\end{enumerate}

Všechny nástroje byly přeloženy do češtiny a zpětně přeloženy do angličtiny pro ověření kvality překladu. HPT dotazník byl navíc zkontrolován nezávislým učitelem dějepisu s dostatečnou praxí a znalostí modelů historického myšlení. Všechny dotazníky níže v sekci Materiály.

\subsection{Výsledky}\label{vuxfdsledky}

\textbf{Hlavní výsledek} -- nepotvrdilo se, že by příklon žáků k autoritářským nebo extremistickým postojům zlepšoval výsledky v HPT testu.

\begin{itemize}
\tightlist
\item
  Vztah mezi postoji žáků a jejich výsledky v HPT byl \textbf{velmi slabý}.
\item
  Žáci s autoritářskými postoji nedosahovali statisticky lepším výsledků, spíše data naznačovala opak (i když velmi slabě).
\end{itemize}

\textbf{Závěr?} Výsledky v HPT testu tedy skutečně odrážejí schopnost žáků zaujmout dobové perspektivy, nikoliv jejich současné názory. To je dobrá zpráva pro učitele -- můžete HPT test použít pro měření, jestli žák dokáže zaujmout dobovou perspektivu a nemusíte se moc bát, že by žáci \enquote{ošidili} test svými postoji.

\subsubsection{Podrobnější výsledky}\label{podrobnux11bjux161uxed-vuxfdsledky}

Rozložení výsledků HPT zhruba odpovídá normálnímu rozložení, které se dalo očekávat. Celkově žáci dosahovali o něco nižších výsledků než v původní studii \autocite{Huijgen_et_al2017Toward}, ale rozdíl nebyl velký. Vliv na to mělo i to, že na rozdíl od nizozemské studie se výzkumu zúčastnili i žáci 9. ročníků ZŠ, učilišť a středních odborných škol. V nizozemskě studii šlo pouze o žáky z ekvivalentu gymnázií.

Zároveň ale platí, že zapojení čeští učitelé se rekrutovali především z těch, kteří mají zájem o moderní dějiny a výuku dějepisu, takže nelze vyloučit, že jejich žáci mohou být nadprůměrní i v tomto ohledu. Opakuji, že výzkum ale nebyl zaměřen na reprezentativnost, ale jen na ověření hlavní hypotézy.

\scriptsize

\begin{longtable}[t]{rrrrr}
\caption{\label{tab:unnamed-chunk-1}Základní deskriptivní statistiky celkového skóre HPT}\\
\toprule
Průměr & Rozptyl & SD & Minimum & Maximum\\
\midrule
2.83 & 0.24 & 0.49 & 1.67 & 3.89\\
\bottomrule
\end{longtable}

\normalsize

\scriptsize

\begin{center}\includegraphics[width=0.8\linewidth]{judovo-overovani-historickeho-mysleni_files/figure-latex/hpt_histogram-1} \end{center}

\normalsize

Vztah mezi postoji a HPT je znázorněn na následujícím grafu lineární regresní přímkou. Jak je vidět, vztah je velmi slabý. Tento výsledek byl potvrzen složitějšími statistickými modely (lineární smíšené modely s náhodnými efekty pro třídy a školy), které ale zde neuvádím pro přehlednost. Zájemcům rád poskytnu kompletní analýzy, v budoucnu se snad objeví i v nějakém odborném článku.

Ve vztahu k testu znalostí se potvrdily výsledky předchozích studií, že znalosti se na výsledcích HPT podílejí zhruba z 20 \%. Žáci s lepšími znalostmi dosahovali i lepších výsledků v HPT testu.

\scriptsize

\begin{center}\includegraphics[width=0.8\linewidth]{judovo-overovani-historickeho-mysleni_files/figure-latex/hpt_results-1} \end{center}

\normalsize

\subsubsection{Korelační matice}\label{korelaux10dnuxed-matice}

\scriptsize

\begin{longtable}[t]{l|r|r|r|r|r|r|r}
\caption{\label{tab:unnamed-chunk-2}Korelační matice (Pearson r; pairwise complete.obs)}\\
\hline
  & HPT (celkem) & HPT (bez ROA) & POP* & ROA & CONT & FR-LF mini & KN\\
\hline
HPT (celkem) & 1.00 & 0.90 & 0.64 & 0.72 & 0.78 & -0.03 & 0.37\\
\hline
HPT (bez ROA) & 0.90 & 1.00 & 0.76 & 0.35 & 0.82 & -0.04 & 0.33\\
\hline
POP* & 0.64 & 0.76 & 1.00 & 0.16 & 0.25 & -0.09 & 0.31\\
\hline
ROA & 0.72 & 0.35 & 0.16 & 1.00 & 0.38 & 0.00 & 0.28\\
\hline
CONT & 0.78 & 0.82 & 0.25 & 0.38 & 1.00 & 0.02 & 0.23\\
\hline
FR-LF mini & -0.03 & -0.04 & -0.09 & 0.00 & 0.02 & 1.00 & -0.11\\
\hline
KN & 0.37 & 0.33 & 0.31 & 0.28 & 0.23 & -0.11 & 1.00\\
\hline
\multicolumn{8}{l}{\rule{0pt}{1em}Pozn.: POP* = absence presentistického (populistického) uvažování; ROA = role aktéra; CONT = kontextualizace. FR-LF mini je průměr RD1–RD3 a NS1–NS3. KN = mini test znalostí. Korelace jsou pouze deskriptivní (bez korekce na shlukování ve třídách).}\\
\end{longtable}

\normalsize

Konkrétně byly provedeny následující analýzy:

\begin{itemize}
\tightlist
\item
  analýza spolehlivosti jednotlivých škál (Cronbachova alfa, McDonaldovo omega)
\item
  modely zohledňující shlukování ve třídách a školách (lineární smíšené modely)
\item
  kontrola vlivu znalostí (mini test znalostí)
\item
  test férovosti jednotlivých položek (DIF analýza)
\item
  srovnání struktury testu mezi různými skupinami (vícefaktorová analýza)
\end{itemize}

Tyto analýzy vedly ke stejným závṙům jako jednoduché grafy výše, navíc potvdily robustnost a použitelnost HPT dotazníku v českém překladu a kontextu.

\subsection{Co si z výsledků může učitel odnést?}\label{co-si-z-vuxfdsledkux16f-mux16fux17ee-uux10ditel-odnuxe9st}

\begin{enumerate}
\def\labelenumi{\arabic{enumi}.}
\tightlist
\item
  \textbf{HPT dotazník lze používat pro ověření schopnosti žáků zaujmout dobové perspektivy.} Samozřejmě je vhodné ho kombinovat s dalšími nástroji a přístupy (např. HistoryLabovými cvičeními zaměřenými na tento koncept), ale jako jeden z nástrojů pro diagnostiku historického myšlení je použitelný.
\item
  \textbf{Schopnost dobových perspektiv není to samé jako souhlas s danými postoji.} I žáci s autoritářskými nebo pro-nacistickými postoji mohou mít potíže s porozuměním lidem v minulosti v jejich dobovém kontextu. Naopak žáci s demokratickými postoji nemusí mít automaticky dobové perspektivy zvládnuté.
\item
  \textbf{Znalosti jsou podstatné, ale nejsou vším.} Znalosti o daném historickém období (v tomto případě meziválečné Německo) hrají roli v tom, jak dobře žák dokáže zaujmout dobovou perspektivu, ale nejsou jediným faktorem a zdaleka ne tím hlavním. Je důležité rozvíjet i další dovednosti a schopnosti historického myšlení.
\end{enumerate}

\section{Materiály}\label{materiuxe1ly}

\begin{itemize}
\tightlist
\item
  \href{https://docs.google.com/document/d/1dVALx38nJEfqsNkWrgBRTvNikQI5fTJ3-S9JJCXsd5Q/edit?usp=sharing}{Stručný výzkumný záměr (anglicky)}
\item
  \href{https://docs.google.com/document/d/10geollT853sY61kzo2I76jjGK4XT5bYbCAtHSZfF6s8/edit?usp=sharing}{Mini test znalostí} (není určen pro sdílení se žáky)
\item
  \href{https://docs.google.com/document/d/1MdAjadyAtGVR-gfEgizH5-RXEBgc0I-cpo3qNG2VAtY/edit?usp=sharing}{Překlad HPT dotazníku do češtiny} (není určen k sdílení se žáky)
\item
  \href{https://docs.google.com/document/d/1qkEgybQVgHTRuyUbPhlDYg0_SjQLAxq9gkS4jzUj5e8/edit?usp=sharing}{Adaptace FR-LF + KSA + SDR dotazníku do češtiny} (není určeno ke sdílení se žáky)
\item
  \href{https://docs.google.com/forms/d/1cL_q5lPQ6lpoHRFE_gyXv6z8oWWou6jrpHVk1lYWTsQ/copy}{Google Form šablona celého dotazníku ke kopírování} (vyžaduje Google účet)

  \begin{itemize}
  \tightlist
  \item
    po vytvoření kopie \textbf{je určeno k online zadávání žákům}
  \end{itemize}
\item
  \href{https://docs.google.com/document/d/1oEAfcp7fBjfRQW56GQSGQ9JG2QocuZivh3VRTWBxK6U/edit?usp=sharing}{Papírová verze k tisku (Google Docs)} - \textbf{je určeno k zadávání žákům}

  \begin{itemize}
  \tightlist
  \item
    \href{https://drive.google.com/file/d/1I2cWLWs26P0-rOVmPBtnIV9M_F-htyk3/view?usp=sharing}{Papírová verze k tisku (PDF, verze na 1 list)}
  \item
    \href{https://drive.google.com/file/d/1IZKhkWbS8PiEFHFpI_FTiZxQ3RfPY8Vw/view?usp=sharing}{Papírová verze k tisku (PDF, verze na 2 listy)}
  \end{itemize}
\item
  \href{data/029-hpt-a-extremismus.RDS}{Anonymizovaný dataset pro vlastní analýzy (RDS)} (\href{data/029-hpt-a-extremismus-codebook.pdf}{codebook})
\item
  \href{data/029-hpt-a-extremismus-report.pdf}{Ukázka výsledného reportu pro učitele (PDF)}
\end{itemize}

\chapter{Umí žáci vyvozovat oprávněné závěry z pramenů?}\label{umuxed-ux17euxe1ci-vyvozovat-opruxe1vnux11bnuxe9-zuxe1vux11bry-z-pramenux16f}

\emph{Argumentuj na základě pramene\ldots{} Opravdu to pramen říká? \ldots{} Není to už jen interpretace?}

V hodinách dějepisu a zejména v badatelské výuce po žácích často chceme, aby vyvozovali své závěry z pramenů. Jenže není vždy snadné poznat, jestli to skutečne umí - schopnost vyvozovat oprávněné závěry z pramenů se ztrácí pod vrstvou dalších dovedností a znalostí, jako je obecné čtení s porozuměním, schopnost pracovat s historickými prameny, schopnost formulovat závěry a argumenty, atd.

Tento výzkum si klade za cíl vytvořit nástroj, který by nám pomohl zjistit, jak žáci závěry z pramenů vyvozují, aniž by v tom hrály vliv jiné faktory.

Nejde o test faktů.

Nejde o zkoušení znalostí učiva.

Jde o to, jak žáci uvažují nad zdroji a tvrzeními o minulosti.

\section{Proč by to pro učitele mohlo být užitečné?}\label{proux10d-by-to-pro-uux10ditele-mohlo-buxfdt-uux17eiteux10dnuxe9}

Zapojeným učitelům může výzkum přinést:

\begin{itemize}
\tightlist
\item
  přehled o tom, jak jejich žáci pracují s tvrzeními a důkazy,
\item
  srovnání různých tříd (pokud se zapojí více tříd),
\item
  podněty, kde mají žáci tendenci „přeskakovat`` k rychlým závěrům,
\item
  konkrétní doporučení, jak tyto dovednosti dál rozvíjet.
\end{itemize}

Výsledkem nebude jen tabulka bodů, ale stručný a srozumitelný report pro Vaši třídu, který ukáže:

\begin{itemize}
\tightlist
\item
  v čem jsou žáci silní,
\item
  kde mají typické potíže,
\item
  jaké typy úvah se u nich objevují.
\end{itemize}

Cílem není hodnotit učitele ani školu.
Smyslem je nabídnout nástroj, který může pomoci lépe zacílit výuku.

\section{Jak bude výzkum probíhat?}\label{jak-bude-vuxfdzkum-probuxedhat}

\begin{itemize}
\tightlist
\item
  \textbf{Kdo?} 9. ročník ZŠ a všechny ročníky SŠ
\item
  \textbf{Čas:} cca 30--45 minut
\item
  \textbf{Jak?} Online test, který žáci vyplní na počítači, tabletu nebo telefonu.

  \begin{enumerate}
  \def\labelenumi{\arabic{enumi}.}
  \tightlist
  \item
    Obdržíte unikátní odkaz (a QR kód).
  \item
    Odkaz pošlete žákům nebo promítnete ve třídě.
  \item
    Žáci test vyplní online.
  \item
    Není potřeba žádná registrace.
  \end{enumerate}
\item
  \textbf{Co žáci budou vyplňovat:}

  \begin{enumerate}
  \def\labelenumi{\arabic{enumi}.}
  \tightlist
  \item
    Krátké základní údaje (anonymně).
  \item
    Úkoly se zdroji a tvrzeními -- vybírají, zda je tvrzení oprávněné, a zdůvodňují proč.
  \item
    Dvě složitější situace z „reálnějšího`` kontextu.
  \end{enumerate}
\end{itemize}

Účast žáků je dobrovolná a anonymní.

\subsection{Ukázka úkolu}\label{ukuxe1zka-uxfakolu}

\begin{figure}
\centering
\includegraphics{./images/030-ukazka-testove-ulohy.png}
\caption{Ukázka úkolu}
\end{figure}

\section{Co jako zapojený učitel dostanu?}\label{co-jako-zapojenuxfd-uux10ditel-dostanu}

\begin{enumerate}
\def\labelenumi{\arabic{enumi}.}
\tightlist
\item
  Souhrnný report za Vaši třídu

  \begin{itemize}
  \tightlist
  \item
    jak žáci pracují s tvrzeními a důkazy,
  \item
    kde mají tendenci dělat ukvapené závěry,
  \item
    jaký je vztah mezi znalostmi a způsobem uvažování.
  \end{itemize}
\item
  Doporučení do výuky

  \begin{itemize}
  \tightlist
  \item
    tipy na úpravu zadání úloh,
  \item
    návrhy, jak pracovat s tvrzeními a důkazy,
  \item
    inspiraci pro vlastní aktivity.
  \end{itemize}
\item
  Možnost konzultace výsledků (online / e-mailem).
\end{enumerate}

Kromě toho dobrý pocit z toho, že pomůžete ověřit nástroj, který může být užitečný pro širší komunitu učitelů. Podílíte se na výzkumu, který má ambici zlepšit způsob, jak ověřujeme práci žáků v dějepise.

\subsection{K čemu to může být ve výuce?}\label{k-ux10demu-to-mux16fux17ee-buxfdt-ve-vuxfduce}

Možná znáte situace, kdy:

\begin{itemize}
\tightlist
\item
  žáci rychle odpovídají, ale jejich závěry nejsou úplně opřené o zdroj,
\item
  část si třídy je velmi jistá, i když důkazy nejsou silné,
\item
  někteří žáci jsou opatrní, ale neumí to vysvětlit.
\end{itemize}

Test může pomoci odhalit právě tyto rozdíly.

Ukáže, jestli problém spočívá spíš v práci s důkazy, v příliš rychlém zobecňování, nebo v nejistotě při formulování závěrů.

\chapter*{Literatura}\label{literatura}
\addcontentsline{toc}{chapter}{Literatura}


% Bibliography
%%%%%%%%%%%%%%%%%%%%%%%%%%%%%%%%%%%%%%%%%%%%%%%%%%%%%%%%%%

\backmatter
\SmallMargins

\twocolumn
\renewcommand*{\bibfont}{\scriptsize}
\printbibliography
\onecolumn


% Tables (of tables, of figures)
%%%%%%%%%%%%%%%%%%%%%%%%%%%%%%%%%%%%%%%%%%%%%%%%%%%%%%%%%%


\cleardoublepage
\LargeMargins
\listoffigures


% After-body (LaTeX code inclusion)
%%%%%%%%%%%%%%%%%%%%%%%%%%%%%%%%%%%%%%%%%%%%%%%%%%%%%%%%%%




% Back cover
%%%%%%%%%%%%%%%%%%%%%%%%%%%%%%%%%%%%%%%%%%%%%%%%%%%%%%%%%%%

% Even page, small margins, no running head, no page number.
\evenpage
\SmallMargins
\thispagestyle{empty}

\begin{normalsize}

\begin{description}

\selectlanguage{english}
\item[Abstract]
English abstract, on the last page.

This is a bookdown template based on LaTeX memoir class.
\item[Keywords]
Keyword in English, As a list.
~\\

\end{description}

\end{normalsize}

\vspace*{\fill}
\centering\includegraphics[width=.3\textwidth]{images/logo}

\end{document}
